\documentclass{article}
\usepackage[utf8]{inputenc}
\usepackage{graphicx}

% change reference style to [1], remove stupid sorting, language changed so date in ddmmyyyy
\usepackage[backend=biber, style=numeric, sorting=none, language=australian]{biblatex}
\addbibresource{References.bib}

\title{Dissertation - Detecting User Engagement Using Mouse Tracking Data}
\author{David Saunders (910995)}
\date{September 2020}

\begin{document}
\maketitle

\begin{abstract} 
    Write abstract here
\end{abstract}

\tableofcontents

% Toms layout

% Motivation
% Related Work
% Implementation
% Results
% Conclusion

\section{Motivation}

% Why is it important
% Why is it a hard problem/problems
% Why other authors have failed
% My contributions are - bullet list
% My idea to fix it

Detecting use engagement is an important science across multiple disciplines.

Find a study that says during any task / any crowdsourced task X\% of people are not paying attention.
This could jeopardise the results from any online if we cannot be sure users are actually paying attention to the task they are being paid to complete. 

It is a hard problem because, among other things I will touch on, it is hard to quantify exactly what user engagement is exactly.
One author defined it as 'XYZ' [look at my previous work for references].

However the previous authors have failed to detect and measure user engagement as X,Y,Z.

\subsection{Contributions}
% The contributions are 1,2,3 (everyone likes 3s)
% Take aways of how it will help them.

In this project my contributions to fix it are, -a system to classify users, a way of visualising their mouse paths, ways to directly and quantitatively compare different users, and a multistage semi-supervised based binary classification output to answer the question of 'are users engaged'.

\section{Related Work}

% Convince the reader you've done enough work and research in the area.
% Convince them it was hard and the topic is hard so you need to help.
% You've tried other peoples techniques.

All related work about actual other attempts to detect user engagement from mouse tracking data.
Other subsections will be like miniature literature reviews or something?

\subsection{Semi Supervised Learning}
This will be a key aspect of the project as we only have definitive labels for part of our data.
This reflects the challenges of real world data, by some estimates only 2\% (made up number) of all data is structured and labelled, the rest is unstructured [find reference].

In a sense the data here is labelled.
All day belongs to 2 classes, online turk user, or in person lab study user.
However on the other hand the data is not labelled for the task I would like to explore.
The goal of this project is to identity which users were paying attention.
We have assumed that we can infer that lab study users will be paying attention, so we can say that those samples are labelled.
However the rest of the samples we have which are all of the online data are unlabelled.
Therefore this is a kind of semi supervised learning problem where we only have a small percentage of our samples labelled, and only confident labels for one class.
(See if a paper on this exists, semi supervised binary classification with only labels for one class.)
% Do a literature review type of thing where I review semi supervised learning, talk about methods and how they can be applied to what I'm trying to do.

\subsection{N-Grams}
This paper has a nice scientific explanation of n-grams \cite{tomovic2006n}.
Either cite this paper or more likely look at their reference for n-grams and cite that.

\subsection{Hidden Markov Models}

% explanation from scratch: https://towardsdatascience.com/hidden-markov-model-implemented-from-scratch-72865bda430e

Paper Tom send me about HMM for text classification that I might be able to use \cite{collins2016tagging}.

Paper claims to use HMM for spam detection, I think they actually just use it to detect misspellings of words or something which is used by spam to hide from filters \cite{gordillo2007hmm}.
Probably could find a better spam detecting HMM, I just like how this almost does something different from the title, which my diss will end up doing.

This paper was recommended by someone on stack overflow as an old influential paper in the field with tens of thousands of references \cite{rabiner1989tutorial}.
Called a tutorial on HMM and its so old so original source. 
Would definitely be good to reference if I include any of the mathematics behind HMMs. 

Someone's dissertation on the topic of generating synthetic data with HMMs.
This can be a good way to create synthetic data \cite{ferrando2018generating}.

Maybe this would belong under implementation?
%HMM-Example-Diagram.png
\begin{figure}[ht]
    \centering
    \includegraphics[scale=0.55]{HMM-Example-Diagram.png}
    \caption{Diagram of a Markov Chain model representing a users mouse position}
    \label{fig:Markov}
\end{figure}

Figure \ref{fig:Markov} shows the states of a markov model of my system with states A-E representing sliders 1-5 and state F representing an html element.
In the top right we can see a possible transition matrix of our system.

\section{Implementation}

% Talk about all different steps you took?

\section{Methodology}

%(1/2 page to 1 page)
%(maybe in methodology)
 The assumption of this project is that labs are paying attention and turks are not. Therefore if we get any outliers from the turk data, but that are similar to the lab data, then we will say that that online turk user was paying more attention than his peers, and that they were paying attention.

This study (ref) says that only 10\% of all people / turk users pay attention during a task.
We will look at the 10\% (30ish) of the turk data that looks like it is lab data and day that they were paying attention.
This is just the assumption we have made for the project, 
unfortunately the dataset isn't extensive enough for us to fully test this hypothesis.

\section{Data Pipeline}

When planning and completing this project many decisions were made about the steps taken to convert the raw data to a finished product/classification.
This section may act as an overview of the project, detailing the different sections of work, what they may contain, and the order in which they will be completed.

\begin{figure}[ht]
    \centering
    \includegraphics[scale=0.55]{Images/Data-Pipeline.png}
    \caption{Diagram of the Data Pipeline of the project.}
    \label{fig:test}
\end{figure}

% Everyone is a question

\subsection{Data}

% Duplicate or remove to deal with imbalances.
% Probably less effective focusing on this rather than features.

Here I will summarise what I have done to the data.
As previously mentioned the data was gathered through a lab study and an online study.
The purpose of the online study was to gather a larger amount of data that was possible to do so in person.
The aim of this was to make any results more statically significant.
Data was recorded in a big JSON dump with lots of irrelevant and repealed data relating to the users background and not their mouse movements.
Pythons JSON couldn't directly convert the data as mouse events were stored as a nested JSON dictionary and there were errors in the way it was written making it invalid JSON.
Took ages and should explain more indepth, but then finally got data into a tabular data format of a csv which I am more comfortable working with.
Ended up with over 100,000 lines? Might have been more. 

This leads to the problem of imbalanced data samples.
There were approximately 11 lab data and 400 online data, meaning that there were 40x as many data samples from one class compared to the other.
As stated in my assumptions, we can say that the lab participants were paying attention, where as the online participants may or may not have been paying attention.

If the classes were balanced then simple approach may be to treat this problem as binary classification problem.
Using something like a Support Vector Machine we could classify a given point as lab or online / paying attention or possibly paying attention based on their proximity to other data points.
To do so we would need to have balanced classes otherwise the algorithm would have a high accuracy from just classifying everything as possibly paying attention as that is the most frequent class.
There are two main methods of dealing with class imbalances, removing data samples and creating new data samples.

It was decided that creating new data samples would be best as there is not a whole lot of data to work with, so there would be a strong preference to keep the data we have.
New points can be created by sampling from a distribution (reference) but here it was decided just to duplicate the samples as there was no discernible distribution of the data.
Another method is to copy the points, altering them slightly, this was considered but not used in the end.
Each lab study data sample was copied 40 times to even out the classes.

\subsection{Features}

% Tom said something about this being where I should spend the most time of my project.
% Spending time here will be more time effective than other sections.
% This is what people wanna see?
% Evaluating effectiveness of different features.

% What do I so to the raw data.

% What are good features? Big/Good question split into smaller sections.

This part of the pipeline refers to what features I am going to extract from the data.
Features of data can be defined as 'attributes or interesting things from the data'. [reference]
These will consist of both raw and created features, but what do I mean by this?
Raw features will consist of the the number of mouse events recorded, while a created feature could be comparing the trace of users cursor data when using the program.

\subsection{Machine Learning}

% What ML will I do and what algorithm.
% MAybe clustering algorithm but on what data?

Once we have insightful features from the data we can consider what machine learning algorithm would be most appropriate to use on the data.
This will obviously be highly dependent on what form the final features are in.
For example if the features are numerical values such as time taken to complete task and number of mouse events then an algorithm such as a Support Vector Machine would be a good choice.
If the data is in the form of sequential data such as a list of all mouse events then something like a LSTM or RNN network would be best suited.
If the features output was an image such as a trace of mouse position over time then a CNN could be a good choice as they're designed for image data.
It is likely that text classification algorithms will be used when comparing the targets of mouse events.
Comparing n-grams can be done with algorithms such as XYZ [reference].
Other text classification algorithms such as cosine similarity or sentiment analysis could also be used.


\subsection{Labels}

% Who's paying attention

Lastly an important section of the pipeline, as it reflects the final outputs of the system.
Labels will refer to which users are classified as paying attention and which users are not.
A key aspect of this project will be semi-supervised learning.
That is we have some data points we can confirm were paying attention, and others where they're level of attention was questionable.
Once we have an algorithm that can classify some users as paying attention or not we can rerun the algorithm with these preliminary outputs as new training data.
If this is done recursively then we can end up with a system that can split all data points into the 2 classes, perhaps with a degree of confidence given as a percentage.


\section{Repeated Experiments}
% Suggested by Tom as an overview of all the work I've done. 
% Lots of it will probably need adjusting and deleting.

This section will give an overview of the work I have completed during this project.
Each sub section offers a unique approach to tackling the problem.
The different methods were tried sequentially, and the pipeline had to be amended for each one as the input requirements changed.

Approach of this project was to try many different methods of looking through, classifying, and predicting attentiveness of this data.

(Get the actual exploration and ml/data science steps by looking at notebooks.)

\subsection{Distance Based Methods / Spacial Classification Methods}

Not sure if this is the best name for it.
These experiments attempted to separate the data samples for turkers, and lab participants by plotting samples in space.
The aim is that this might reveal information about the features of different users.
If there was a clear distinction between the classes then we could potentially reclassify some points based on their euclidean distance to one another.

\subsubsection{SVM separation methods}
Here we look at if a simple hyperplane could separate the data when looking at similar features the number of mouse events and the total time taken to do the task. 
It was found that a linear, or hyperbolic(wasn't hyperplane was something else) was unable to separate data. 
This was mainly die to the unbalanced classes.

Here simple features of the data was used, the time taken to complete the task, and the number of mouse event taken to complete the task.


\subsubsection{Clustering}
Didn't do this but its an example point.
(Might have tried kNN).

\subsection{Bag-Of-Words Methods}

(Maybe mention bag-of-words models in background research?)

Can treat each target as a word as use NLP methods such as bag-of-words. 
With with method we will lose data regarding order but that can be analysed later with HMMs.
"In this model, a text (such as a sentence or a document) is represented as the bag (multiset) of its words,
 disregarding grammar and even word order but keeping multiplicity.
  The bag-of-words model has also been used for computer vision." 
  - ($wikipedia.org/wiki/Bag-of-words_model$)

Disregarding grammar is a big weakness of traditional uses of BoW but the concept of grammar is meaningless in the case of a sequence of mouse targets.

\subsubsection{Naive Bayes}

Another idea from SPAM detection is using a naive bayes, here they get the count of each word for each email and bayes it. 
 https://towardsdatascience.com/spam-filtering-using-naive-bayes-98a341224038

Not very good, when balanced data used cross validation average of 50ish%.

https://machinelearningmastery.com/failure-of-accuracy-for-imbalanced-class-distributions/
This website has some nice references about imbalanced class distributions.
( TODO: use in diss)
data sampling – customized algorithms– cost sensitive algorithms– one class algorithms – threshold moving – probability calibration


$https://machinelearningmastery.com/tour-of-evaluation-metrics-for-imbalanced-classification/$
Better ways of dealing with imbalenced data

\subsubsection{Naive Bayes with N-Grams}
Could even use this to get the counts of 2/3-grams to compare?
More successful but suffered from same issues.

Include image of n-grams distributions?
TODO: Plot of ngrams for both classes. Create side by side bar plot.

html seems to be higher with turkers rather than lab users, this may be because they were fidgeting and not paying attention?

\subsection{Hidden Markov Models}
% Useful Background reading and potential references.
%https://towardsdatascience.com/hidden-markov-models-for-time-series-classification-basic-overview-a59b74e5e65b Normally towardsdatascience is really good but this makes no sense to me at all.

%This article is more complicated, but also indepth and explains stuff more. - https://towardsdatascience.com/markov-chains-and-hmms-ceaf2c854788
%Some guys github repository using HMM for binary sentimental analysis of tweets - https://github.com/FantacherJOY/Hidden-Markov-Model-for-NLP
%"In the case one label is way more frequent than the other (say Class B appears for every 100 samples of Class A), or in the case one category is broader than the other (have more variance, i.e. Class A is an expected behavior of a process and Class B is all the cases that are abnormal), then we might want to train a unique HMM for Class A. For each new sample, we compute the likelihood to the model, and if it falls below a pre-defined threshold (that can be chosen with respect to some validation sample, or in the worst case, from the training samples themselves) we assign this sample to Class B. If the likelihood is above the threshold, we assign the sample to Class A." - https://datascience.stackexchange.com/questions/8560/can-hmm-be-used-as-a-binary-classifier

HMMs are typically used with time series data.
Here I just used the series of mouse events and ignored the timestamps of the event.
It was found earlier that the length of mouse events and time are positively correlatied with a pearsons correlation of ?? 0.6,
 and so just the sequence of mouse events should still hold the detail that time data would.

These can be used for binary classification as described here 
(https://datascience.stackexchange.com/questions/8560/can-hmm-be-used-as-a-binary-classifier). Say theyre used to generate samples, maybe in another part as that will probably need more explanation.

With the bag of words model we ignored all order to the sequences looking only at the frequency of each item.
This is a different approach, here the order is the most important thing.

EXPLAIN EVERYTHING IVE DONE.

\subsection{Imbalanced classes}

HMMs can be used for classification as shown above, but they are also known as generative models.
This means they can be used to create new data.
The above methods of SVM, KNN, and Naive Bayes failed due to the imbalances of classes.
Now the trained HMMs could be used to generate new data points, to reduce the imbalances, and make these methods effective.

\section{Results}

% talk about results
% Should I talk about all the results, or just the end results that I feel are conclusive.

\subsection{Table or graph of results}

\section{Conclusion}

To evaluate the results I would not be too confident in my findings.
The data was incorrectly labelled for the task I wanted to perform.
Additionally the data was heavily imbalanced.
All of the conclusions I have made from the lab data are relying on only a handful (13 maybe) of datapoints which is not statistically significant.
Additionally there was not a lot of data from the other class of online data either.
Typically data science and machine learning uses big data, where as this project used only 400ish records in total which came to only 500mb of data (MAYBE).

Biggest flaw of the project is the first assumption, that the lab people were all paying attention.
While they were monitored and we can be sure they were not distracted by phones or tellevisions, many of them may have 'zoned out' and not have been given it their full effort and attention.

\section{Future work}
It would be interesting to see if any of the methods here would reveal anything interesting in other datasets.
A kaggle crowdflower dataset seems like an idea candidate and I would be interested if we could determin if any of these users were paying attention. 

\printbibliography

\end{document}